\documentclass{beamer}

\usepackage{default}
\usepackage{xeCJK}
\usepackage{fontspec}
\setmainfont{SimSun}
\usetheme{DarkConsole}

\title{研究报告}
\author{陈泰然}
\institute{USTC nelslip}
\logo{\includegraphics[width=2cm,height=2cm]{figures/ustc_logo_fig}}
%\date{}

\begin{document}
\frame{\titlepage}

%\begin{frame}
%	\frametitle{目录}
%	\tableofcontents
%\end{frame}

\AtBeginSection[]
{
	\begin{frame}
		\frametitle{目录}
		\tableofcontents
	\end{frame}
}

\section{Dynamic Programming}

\begin{frame}
	\frametitle{Dynamic Programming}
	RL中的Dynamic Programming(DP)指的是一类算法,给定一个完美的环境模型,DP算法可以计算其对应的最佳策略。\\
	这里先假设环境是一个finite MDP,该MDP的“完美”性体现在其一步变换转移概率是完全已知的,即$p(s^{'},r|s,a)$已知。\\
	DP的主要思想,是利用value function来组织策略的搜索。 DP通过将Bellman Optimality Equation转化为能够不断优化value function近似值的更新规则。
\end{frame}

\begin{frame}
	\frametitle{Dynamic Programming}
	DP算法由两部分组成:
	\begin{itemize}
		\item 先考虑对任意的策略$\pi$,如何计算$v_\pi$。这一步叫做策略估计(Policy Evaluation)。
		\item 再考虑如何修改策略$\pi$使$v_\pi$变大。这一步叫做策略优化(Policy Improvement)。
	\end{itemize}
	策略优化过程在策略为确定性策略的时候,可以推导出策略优化定理:
	令$\pi$,$\pi'$是一对确定的策略,如果对于$\forall s\in S$,有$q_\pi (s,\pi'(s))\ge v_\pi (s)$,那么$\pi'$一定不会劣于$\pi$,也就是说,对于$\forall s\in S$,有$v_(\pi^{'} ) (s)\ge v_\pi (s)$。
\end{frame}

\section{World Model的核心组成}

\begin{frame}
	\frametitle{World Model的核心组成}
	World Model由三部分组成:
	\begin{itemize}
		\item[V] VAE Model
		\item[M] MDN-RNN Model
		\item[C] Controller Model
	\end{itemize}
\end{frame}

\begin{frame}
	\frametitle{V Model}
	在World Model中,V的作用是学习输入图像的一个抽象的、高度压缩的表示。在实验中,采用一个简单的Variational Autoencoder(VAE)作为V模型,来将每一帧输入图像编码成为一个很小的向量$z$。
\end{frame}

\begin{frame}
	\frametitle{M Model}
	M是一个预测模型,它的作用是预测V模型下一步即将产生的$z$。\\
	在实验中,采用一个MDN-RNN模型,其构造为一个RNN加上一个混合密度网络(MDN),用于输出$z$的概率分布$p(z)$。
\end{frame}

\begin{frame}
	\frametitle{C Model}
	C的作用是决定agent下一步的动作,让累积回报期望最大。\\
	在实验中,有意地把C做的非常简单,这样做是为了让agent的复杂度集中在V和M中。C是如下的一个线性模型:
	\begin{equation}
		a_t=W_c[z_t\ h_t]+b_c
	\end{equation}
	在该模型中,$W_c$和$b_c$分别是权重矩阵和偏置向量,将$z_t$和$h_t$连接而成的向量映射到动作向量$a_t$。
\end{frame}

\section{MCTS在World Model中的应用}

\begin{frame}
	\frametitle{World Model+MCTS}
	MCTS的思路是使用随机模拟的方法来选择最佳的动作的,这个思路可以应用到World Model中的Controller模型上。\\
	原来的C Model使用的是一个简单的线性映射模型,将$z_t$和$h_t$拼接成的向量直接映射到动作。\\
	改成用MCTS的思路之后,会展开一个Monte Carlo树,评估每个叶子的累计reward,然后在根节点处取一个使之下的所有叶子的平均累计reward最大的一步动作,然后执行这个动作。
	这个思路现在还没有具体算法实现,有待进一步验证。
\end{frame}

\end{document}
