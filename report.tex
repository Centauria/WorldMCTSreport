\documentclass{beamer}

\usepackage{default}
\usepackage{xeCJK}
\usepackage{fontspec}
\setmainfont{SimSun}
\usetheme{DarkConsole}

\title{研究报告}
\author{陈泰然}
\institute{USTC nelslip}
\logo{\includegraphics[width=2cm,height=2cm]{figures/ustc_logo_fig}}
%\date{}

\begin{document}
\frame{\titlepage}

%\begin{frame}
%	\frametitle{目录}
%	\tableofcontents
%\end{frame}

\AtBeginSection[]
{
	\begin{frame}
		\frametitle{目录}
		\tableofcontents
	\end{frame}
}

\section{World Model的核心组成}

\begin{frame}
	\frametitle{World Model的核心组成}
	World Model由三部分组成:
	\begin{itemize}
		\item[V] VAE Model
		\item[M] MDN-RNN Model
		\item[C] Controller Model
	\end{itemize}
\end{frame}

\begin{frame}
	\frametitle{V Model}
	在World Model中,V的作用是学习输入图像的一个抽象的、高度压缩的表示。在实验中,采用一个简单的Variational Autoencoder(VAE)作为V模型,来将每一帧输入图像编码成为一个很小的向量$z$。
\end{frame}

\begin{frame}
	\frametitle{M Model}
	M是一个预测模型,它的作用是预测V模型下一步即将产生的$z$。\\
	在实验中,采用一个MDN-RNN模型
\end{frame}

\begin{frame}
	\frametitle{C Model}
	C的作用是决定agent下一步的动作,让累积回报期望最大。\\
	在实验中,有意地把C做的非常简单,这样做是为了让agent的复杂度集中在V和M中。C是如下的一个线性模型:
	\begin{equation}
		a_t=W_c[z_t\ h_t]+b_c
	\end{equation}
	在该模型中,$W_c$和$b_c$分别是权重矩阵和偏置向量,将$z_t$和$h_t$连接而成的向量映射到动作向量$a_t$。
\end{frame}

\end{document}
